% !TeX root = ../main.tex

\begin{acknowledgements}
四年大学学习生活即将结束了,如果对这四年的大学生活打分,满分十分我会打七分。三分就扣在大二大三的时候太过懒散,没有利用好时间高效学习。但是总的来说四年里我的确收获了很多,这些收获离不开良师益友的陪伴。\par
我本科专业是飞行器设计与工程,现在搞视觉SLAM,也算半个跨专业了。在导师布树辉老师的悉心指导下,克服了
不少困难。借用这次毕业设计的机会,对新学科的知识进行一次充分的整合。经过这次毕业设计,我的能力有了很大的提高,在编程语言的理解,程序架构等方面都有很大的进步。这期间凝聚了很多人的心血,在此我表示由衷的感谢。没有他们的帮助,我将无法顺利完成这次毕业设计。\par
首先要感谢的就是我的导师布树辉教授。我记得大三的时候我还在纠结选流体还是固体方向,但是自从上过布老师的课以后我就对布老师的研究内容产生了浓厚的兴趣。实话说,我很庆幸能在大三下半年接触到了布老师,走上了研究视觉SLAM这条路。布老师是一个学术水平很高的老师,人生经验也很丰富,代码能力和Debug能力都很强。在如何学习,如何提高效率,如何提高代码质量等问题上给我提了不少有用的建议。布老师管理学生宽松中带着严厉,教研室并不强制打卡上下班,完成任务即可。当然,也不应该用到“管理”这个词,布老师跟同学们是亦师亦友的关系,教研室的气氛很融洽。能在布老师这里继续攻读研究生,我觉得很幸运。\par
然后要感谢的就是赵勇师兄,如果说是布老师领我入门,那赵勇师兄就是带我登堂入室的那个人(当然我现在水平也一般,但相比之前提高了很多)。赵勇师兄让我得以窥到C++高深的一面,是赵勇师兄激发了我对C++的兴趣,我也购买了很多C++的经典书籍来阅读。我曾经一度觉得我的进度太慢了,每天心情都很烦躁,但赵勇师兄会鼓励我,告诉我你做的很棒,这让我尤为感激。同时还要感谢王伟师兄给我介绍了TerrainFusion的原理,让我了解到SLAM贴图流程。感谢刘国晨师兄跟我讲解SLAM流程。感谢徐磊师兄和程宇琪师兄给予的大量学习资料。感谢小师姐让教研室更加多彩。\par
同时,我还要感谢一下李随城同学每天早上叫我起床一起去教研室。他研究深度学习方向,跟他交流有时候也有一些新收获。感谢我女朋友马文阁同学每天关心照顾我这么长时间\par
我还要感谢班主任张光老师大学四年对我们的关怀照顾,张老师每次在开班会的时候都会问我们在学业和生活上有没有什么困难,有困难就找张老师。这给我们莫大的慰藉。\par
除了上述老师和同学们以外,我还想感谢我的父母,虽然相隔甚远,他们也没有学术上的帮助,但他们默默地支持与关心着我,是我学习与生活中坚强的后盾,是我在遇到困难时倾诉心情的港湾。\par
最后,再次对那些在论文完成过程中,关心、帮助我的同学和朋友们表示衷心感谢。
\end{acknowledgements}
