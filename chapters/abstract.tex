% !TeX root = ../main.tex

\begin{abstract}
无人飞行器因其体积小,用途广泛,成本低,效费比好,无人员伤亡风险,生存能力强,机动性能好,使用方便等优点,在现代民用方面有广阔的前景,更在现代战争中具有极其重要的作用。目前无人飞行器已广泛运用在航拍,测绘,战场侦查等方面。在未知环境中,无人飞行器往往要求实时获取周围地图信息,同时构建场景地图来满足定位,规避障碍,规划路径和侦查周围环境的要求,而室内环境往往比室外环境更为复杂,具有更加严苛的要求。所以,无人飞行器的定位与周围环境感知在此应用背景下就显得极其重要。\par
本文通过SLAM过程来对机器人进行定位,并根据定位结果进行三维稠密重建周围环境场景。主要内容是基于ORB特征点法的位姿估计研究,基于块匹配的三维点云绘制。\par
首先,本文对SLAM现状进行介绍,然后针对SLAM过程中几个重要的特征点进行了介绍,比如特征点提取,前端设计,后端优化,回环检测等。同时对对极几何,李代数,针孔相机等数学模型进行了系统介绍。\par
最后我们设计实现了一个基于块匹配的三维重建算法,利用SLAM过程得到的位姿和特征点重建出了单张照片的深度,取得了不错的效果。\par
\keywords{SLAM;特征点;对极几何;非线性优化;词袋模型;三维稠密重建}
\end{abstract}

\begin{enabstract}
Unmanned aerial vehicle (UAV), with its small size, wide use, low cost, good efficiency and cost ratio, no casualty risk, strong survivability, good maneuverability, easy to use and so on, has a broad prospect in modern civil field, and has an extremely important role in modern war. At present, UAV has been widely used in aerial photography, surveying and mapping, battlefield detection and so on. In the unknown environment, unmanned aerial vehicles often require real-time access to the surrounding map information, at the same time building a scene map to meet the location, avoid obstacles, plan the path and investigate the surrounding environment requirements, and indoor environment is often more complex than the outdoor environment, with more stringent requirements.
\par
In this paper, the SLAM process is used to locate the robot, and three-dimensional dense reconstruction of the surrounding environment scene is carried out according to the location results. The main content is the research of pose estimation based on ORB method, and the rendering of three-dimensional point cloud based on block matching. \par
Firstly, this paper introduces the current situation of SLAM, and then introduces several important keypoints in SLAM process, such as keypoint extraction, front-end design, back-end optimization, loop detection and so on. At the same time, the mathematical models of polar geometry, Lie algebra and pinhole camera are systematically introduced.\par
Finally, we design and implement a three-dimensional reconstruction algorithm based on block matching, and reconstruct the depth of a single photograph by using the pose and feature obtained from SLAM process, which achieves good results.\par
\enkeywords{SLAM; Key Points; Epipolar Geometry; Nonlinear Optimization; Bag-of-Word; Dense reconstruction}
\end{enabstract}
