% !TeX root = ../main.tex

\begin{abstract}
无人飞行器因其体积小,用途广泛,成本低,效费比好,无人员伤亡风险,生存能力强,机动性能好,使用方便等优点,在现代民用方面有广阔的前景,更在现代战争中具有极其重要的作用。目前无人飞行器已广泛运用在航拍,测绘,战场侦查等方面。在未知环境中,无人飞行器往往要求实时获取周围地图信息,同时构建场景地图来满足定位,规避障碍,规划路径和侦查周围环境的要求,而室内环境往往比室外环境更为复杂,具有更加严苛的要求。所以,无人飞行器的定位与周围环境感知在此应用背景下就显得极其重要。本文主要针对室内场景的定位与场景跟踪进行研究。主要内容是基于 ORB 特征
点法的位姿估计研究,基于 Kinect 的三维点云绘制。首先,本文针对适用于室内空间的并行跟踪与地图构建算法进行了详细介绍,并对 Kinect 相机进行了系统介绍,简述相机的校正和标定。其次,对图像的处理有了详细的介绍,例如特征点提取,前端设计,后端优化,回环检测等。
最后,我们对提取的前面提取的关键帧和地图点绘制在地图中,实现了一个
视觉 SLAM 系统。

  \keywords{SLAM;特征点;对极几何;非线性优化;词袋模型;RANSAC}
\end{abstract}

\begin{enabstract}
  This is a sample document of USTC thesis \LaTeX{} template for bachelor,
  master and doctor. The template is created by zepinglee and seisman, which
  orignate from the template created by ywg. The template meets the
  equirements of USTC theiss writing standards.

  This document will show the usage of basic commands provided by \LaTeX{} and
  some features provided by the template. For more information, please refer to
  the template document ustcthesis.pdf.

  \enkeywords{SLAM; Key Points; Epipolar Geometry; Nonlinear Optimization; Bag-of-Word; RANSAC}
\end{enabstract}
