% !TeX root = ../main.tex
\chapter{对极几何}
通过上一章的介绍,我们知道了如何在一张图中的提取出特征点和对应的描述子,并通过描述子的匹配来获取两张图片特征点的对应关系。在上一章的最后我们提到了基础矩阵和单应矩阵,它们都属于对极几何的范畴,这一章就来介绍一下对极几何。
\section{对极约束}
立体成像的基本几何就是对极几何。图\ref{epipolargeometry}是最经典的对极几何示意图。$O_1$和$O_2$为两个相机(也有可能是一个相机在不同时刻的位置)的位置,P为空间中一点,两个相对的白色平面是归一化平面。$p_1$和$p_2$是P点在归一化平面上的对应点,$e_1$,$e_2$为归一化平面和$O_1$,$O_2$的交点。$O_1$$O_2$为基线,也被称作相机的移动方向。\par
\begin{figure}[H]
	\centering
	\includegraphics[height=5cm]{epipolargeometry}
	\caption{对极几何}
	\label{epipolargeometry}
\end{figure}
这里介绍几个对极几何中常用的概念: $e_1$和$e_2$被称作极点,P$O_1$$O_2$平面为极面,$p_1$$e_1$为极线,同理$p_2$$e_2$也为极线。
所谓对极约束,指的就是相机$O_1,O_2,\text{点}P$在同一个平面上。也就是向量$\overrightarrow{O_1O_2},\overrightarrow{O_1P},\overrightarrow{O_2P}$共面,即三者混合积为0:
\begin{equation}
\overrightarrow{O_1P}\cdot\left[\overrightarrow{O_1O_2}\times\overrightarrow{O_2P} \right]=0
\end{equation}
设相机$O_2$相对于相机$O_1$的旋转为R,位移为t,向量$\overrightarrow{O_1p_1}$为$\vec{x}$($O_1$坐标系下),向量$\overrightarrow{O_2p_2}$为$\vec{x^\prime}$($O_2$坐标系下),两相机的内参是$K_1,K_2$,由相机成像原理知:
\begin{equation}
\begin{aligned}
	d_{1} \boldsymbol{x}_{1}=&\boldsymbol{K}_{1} \boldsymbol{P}\\
	d_{2} \boldsymbol{x}_{2}=&\boldsymbol{K}_{2}(\boldsymbol{R P}+\boldsymbol{t})
\end{aligned}
\end{equation}
联立上式得:
\begin{equation}
d_{2} \boldsymbol{K}_{2}^{-1} \boldsymbol{x}_{2}=d_{1}\boldsymbol{R} \boldsymbol{K}_{1}^{-1} \boldsymbol{x}_{1}+\boldsymbol{t}
\end{equation}
两边同时叉乘t\footnote{$\mathbf{a} \times \mathbf{b}=\left[ \begin{array}{ccc}{0} & {-a_{z}} & {a_{y}} \\ {a_{z}} & {0} & {-a_{x}} \\ {-a_{y}} & {a_{x}} & {0}\end{array}\right] \left[ \begin{array}{l}{b_{x}} \\ {b_{y}} \\ {b_{z}}\end{array}\right]=\left[ \mathbf{a}\right]_{\times}\mathbf{b}$}:
\begin{equation}
d_{2} \left[\boldsymbol{t}\right]_{\times}\boldsymbol{K}_{2}^{-1} \boldsymbol{x}_{2}=d_{1}\left[\boldsymbol{t}\right]_{\times}\boldsymbol{R} \boldsymbol{K}_{1}^{-1} \boldsymbol{x}_{1}
\end{equation}
再同时左乘$\left(\boldsymbol{K}_{2}^{-1} \boldsymbol{x}_{2}\right)^T$:
\begin{equation}
\boldsymbol{x}_{2}^T\boldsymbol{K}_{2}^{-T} \left[\boldsymbol{t}\right]_{\times}\boldsymbol{R} \boldsymbol{K}_{1}^{-1} \boldsymbol{x}_{1}=0
\end{equation}
若令$\boldsymbol{K}_{1}^{-1} \boldsymbol{x}_{1}=\hat{\boldsymbol{x}}_1,\boldsymbol{K}_{2}^{-1} \boldsymbol{x}_{2}=\hat{\boldsymbol{x}}_2$,也就是$\hat{\boldsymbol{x}}_1,\hat{\boldsymbol{x}}_2$分别为归一化平面坐标,则:
\begin{equation}
\hat{\boldsymbol{x}}_2^T\left[\boldsymbol{t}\right]_{\times}\boldsymbol{R}\hat{\boldsymbol{x}}_1=0
\end{equation}
我们称本质矩阵$\boldsymbol{E}=[\boldsymbol{t}]_{\times} \boldsymbol{R}$,基础矩阵$\boldsymbol{F}=\boldsymbol{K}_{2}^{-T} \boldsymbol{E} \boldsymbol{K}_{1}^{-1}$\par
可以看到,我们之所以要求解本质矩阵或者基础矩阵,是因为它们包含着图像之间的运动信息,只要求得本质矩阵,就能够从中分解除R和t,那么相机的运动信息也就清楚了。想要求解本质矩阵,首先要求解基础矩阵,然后去掉内参信息就能得到本质矩阵。求解基础矩阵我们一般有以下方法。
\section{直接线性变换法}
对于一对匹配点$\boldsymbol{x}_{1}=\left[ u_{1},v_{1},1\right]^{\mathrm{T}}, \boldsymbol{x}_{2}=\left[ u_{2},v_{2},1\right]^{\mathrm{T}}$根据对极约束$\boldsymbol{x}_{2}^{T} \boldsymbol{F x}_{1}=\mathbf{0}$:
\begin{equation}
\left( \begin{array}{ccc}{u_{1}} & {v_{1}} & {1}\end{array}\right) \left[ \begin{array}{ccc}{F_{11}} & {F_{12}} & {F_{13}} \\ {F_{21}} & {F_{22}} & {F_{23}} \\ {F_{31}} & {F_{32}} & {F_{33}}\end{array}\right] \left( \begin{array}{l}{u_{2}} \\ {v_{2}} \\ {1}\end{array}\right)=0
\end{equation}
令$f=\left[ F_{11},F_{12},F_{13},F_{21},F_{22},F_{23},F_{31},F_{32},F_{33} \right]^{T}$则有:
\begin{equation}
\left[
{u_{1} u_{1},}{u_{1} v_{2},}{u_{1},}{v_{2} u_{1},}{v_{1} v_{2},}{v_{1},}{u_{2},}{v_{2},}{1}
\right] f=0
\end{equation}
当有n对匹配点时:
\begin{equation}
\boldsymbol{A}=
\left\lbrace
	\begin{array}{c}
	u_{1}^{(1)} u_{1}^{(1)}, u_{1}^{(1)} v_{2}^{(1)}, u_{1}^{(1)},  v_{1}^{(1)} u_{2}^{(1)}, v_{1}^{(1)} v_{2}^{(1)}, v_{1}^{(1)}, u_{2}^{(1)}, v_{2}^{(1)},1\\
	u_{1}^{(2)} u_{1}^{(2)}, u_{1}^{(2)} v_{2}^{(2)}, u_{1}^{(2)},  v_{1}^{(2)} u_{2}^{(2)}, v_{1}^{(2)} v_{2}^{(2)}, v_{1}^{(2)}, u_{2}^{(2)}, v_{2}^{(2)},1\\
	\cdots\\
	u_{1}^{(n)} u_{1}^{(n)}, u_{1}^{(n)} v_{2}^{(n)}, u_{1}^{(n)},  v_{1}^{(n)} u_{2}^{(n)}, v_{1}^{(n)} v_{2}^{(n)}, v_{1}^{(n)}, u_{2}^{(n)}, v_{2}^{(n)},1\\
	\end{array}
\right\rbrace 
\end{equation}
即$\boldsymbol{A}f=0$。
要保证有唯一解至少需要8对匹配点,当匹配点n=8时,若A非奇异,则有唯一解,称为8点法。若n>8,则可用最小二乘解,也就是$\boldsymbol{A}^T\boldsymbol{A}$的最小特征值对应的特征向量即为最优解。
\section{RANSAC法}









