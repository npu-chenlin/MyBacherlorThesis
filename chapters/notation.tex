% !TeX root = ../main.tex

\begin{notation}

  \begin{notationlist}{2em}
  	\item[$\displaystyle \boldsymbol{u}$] 表示像素坐标
  	\item[$\displaystyle \hat{\boldsymbol{x}}$] 表示归一化平面坐标
    \item[$\displaystyle \boldsymbol{P}$] 表示空间点三维坐标
    \item[$\displaystyle \boldsymbol{R}$] 表示一个旋转矩阵
    \item[$\displaystyle \boldsymbol{t}$] 表示一个平移向量
	\item[$\displaystyle \boldsymbol{q}$] 表示一个四元数
	\item[$\displaystyle \boldsymbol{K}$] 相机内参矩阵
	\item[$\displaystyle \boldsymbol{E,F,H}$] 本质矩阵,基础矩阵,单应矩阵
	\item[$\displaystyle{\boldsymbol{a}^{\wedge},\left[\boldsymbol{a}\right]_{x}}$] 表示从三维向量转变为一个3$\times$3的反对称矩阵
	\item[$\displaystyle a \oplus b$] 汉明距离,若a=b取1,反之取0
	\item[$\displaystyle \triangle$] Laplace算子
	\item[$\displaystyle \boldsymbol{I}$] 像素灰度/亮度值
	\item[$\displaystyle \boldsymbol{SE}(3)$] 特殊欧式群
	\item[$\displaystyle \mathfrak{se}(3)$] 李群$\boldsymbol{SE}(3)$对应的李代数
	\item[$\displaystyle \exp(\xi^\wedge)$] 李代数$\xi$的指数映射
	\item[$\displaystyle \boldsymbol{J,H}$] 雅克比矩阵,海塞矩阵
	\item[$\displaystyle \lambda$] 拉格朗日乘子
  \end{notationlist}
\end{notation}



% 也可以使用 nomencl 宏包

% \printnomenclature

% \nomenclature{$\displaystyle a$}{The number of angels per unit are}
% \nomenclature{$\displaystyle N$}{The number of angels per needle point}
% \nomenclature{$\displaystyle A$}{The area of the needle point}
% \nomenclature{$\displaystyle \sigma$}{The total mass of angels per unit area}
% \nomenclature{$\displaystyle m$}{The mass of one angel}
% \nomenclature{$\displaystyle \sum_{i=1}^n a_i$}{The sum of $a_i$}
