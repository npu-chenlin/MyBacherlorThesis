% !TeX root = ../main.tex
\chapter*{毕业设计小结}
这次毕业设计我的初衷是总结一下单目SLAM的流程,每个步骤都梳理一遍,争取做到公式能推导,代码写的出来。但是这一套走完之后发现自己的能力还有欠缺。特别是在非线性优化那一块儿,涉及到矩阵论,群论之类的知识,对李代数扰动模型求导的理解还不够透彻。比如LM算法中引入了拉格朗日算子,我对拉格朗日算子又不是很了解。这些知识每深入一分就会牵扯出更多的知识,由衷的让我觉得我还应该多加学习。但是毕设已经写出来了,我自我感觉还可以。我认为这次的毕设再修修改改就能够当做一个SLAM入门书籍。SLAM说难其实也不难,搞清楚流程,仔细设计代码架构也能写得出一个SLAM框架。我下一步想法是尝试写一写SLAM框架。布老师说的最多的就是我们缺乏练习,只有写出一个框架你才能说真正理解了这个东西,不然都是光说不练假把式。\par
这一年下来,我的C++水平的提高也提高了。这个体现不到毕业设计上,但是是确实有提高的。自己私下也买了好多C++的圣经打算读一读,里面有很多关于如何设计出健壮稳定的建议。这段时间深深地觉得光看不写等于没看,这些好的建议只有在你见过情景的时候才能起到作用,不然光看是没用。赵勇师兄的Svar给我留下了很深的印象,我才知道原来C++还能这么用,里面的模板元大大化简了程序的复杂性,能让人编写出更优美的代码,激起了我学习模板元的兴趣。\par
话又回到SLAM上,当前的SLAM热点是多传感器融合,在BA中添加诸如GPS,IMU等信息,能够使SLAM系统更加健壮,准确率也更高,暑期的时候也应该了解一下这方面的知识。\par
暑期有这么几个想法:
\begin{enumerate}
\item 学习概率论,矩阵论,了解一下最优化算法,看一看西瓜书
\item 尝试写一写简单的SLAM框架,更深入的了解程序是怎么运作的
\item 了解模板元编程,大概理解常见用法
\item 了解SLAM多传感器融合,重点看GPS和IMU。
\end{enumerate}\par
总之就是学习学习再学习,自己的水平还很差,只有不断学习才有可能做出有意义的工作。