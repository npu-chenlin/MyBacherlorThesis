% !TeX root = ../main.tex
\chapter{总结和展望}
本文详细介绍了单目视觉SLAM的理论知识和主要的技术点后,基于DIYSLAM实现了一个视觉SLAM,并在SLAM系统的输出的基础上进行了三维稠密重建。前者主要利用单目相机采集的图像数据集实时快速的恢复相机位姿和三维稀疏结构,后者在相机内外参数已知的基础上,利用块匹配技术找到每个像素点的最优匹配,估算出图像帧中每个像素点对应的深度,完成稠密重建。\par
本文虽然对定位和环境感知做了一定的研究,也取得了不错的结果,但是还有很多局限性。单目相机无法获取场景的真实尺度,得到的都是相对值,对很多应用有限制,如果只是单一的利用视觉信息会对特征有很大的依赖,因此结合其他特定传感器的观测数据提高精度和鲁棒性才能更好的满足实际需求。比如现在主流的视觉-惯导融合方法,相机在快速运动时会出现运动模糊导致相机数据无效,这正好与惯性传感器 (IMU) 的特性互补,因为IMU可以测出传感器在运动时的角速度和加速度。因此,对于 IMU 或者深度摄像头等其他传感器提供的观测数据,如何更好的与视觉 SLAM 融合,使 SLAM系统更加鲁棒。\par
基于图像块匹配的三维重建算法算法,在时间性能上与很多现有稠密重建方法依然有差距。重建一幅图需要5-7秒的时间。但是由于全像素计算深度有一部分操作是和相邻像素无关的,我们就可以考虑用GPU进行加速,同时也可以考虑仅对图像的边缘部分的像素进行重建,可供后期贴图之用。